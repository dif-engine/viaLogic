\documentclass[fleqn]{jsarticle}
\title{論理学からはじめる数学: 第3講}
\author{川井新}
\usepackage{amscd}
\usepackage{amsmath}
\usepackage{amsthm}
\newtheorem{theo}{Theorem}
\newtheorem{defi}{Definition}

\begin{document}
\maketitle

\section{証明の掟}

数学の証明の掟: 証明で用いた条件を定理内で明示せよ

証明図の掟: 木のいちばん上の式 (仮定) はすべて落とされなければならない。


証明図が掟を守っているのか一見、わかりにくい。
そこで仮定の式に文字$x, y, z, \ldots$を用いてラベルを付け、
以下のように規則にラベルの使い方もハンドルする。

%tableOfRulesWithTerm


このとき、
掟を守ったラベルと掟を破ったラベルを性格づけられる。


\section{ラムダ項} 

ラベルの正体は、
ラムダ項である。

%accountsOfLambdaTerm

\begin{defi}
ラムダ項のフォーマルな定義は以下で与えられる:
  \begin{enumerate}
    \item 変数$x, y, z \ldots$はラムダ項である。
    \item $M$と$N$がラムダ項なら、$(MN)$はラムダ項である (「適用」) 。
    \item $M$がラムダ項で$x$が変数なら、$(\lambda x. M)$はラムダ項である (「抽象」) 。
    \item 以上でわかるものだけがラムダ項である。
  \end{enumerate}
\end{defi}

この定義では、 (1) でもっとも簡単なラムダ項を与え、
(2) - (3) でこれらから新しいラムダ項を作り出す規則を与えている。
このような定義を帰納的定義という。

適用 (application) が除去規則に、
抽象 (abstraction) が導入規則に対応する。

ある表現のなかに同じ記号が複数回現れるとき、
その現れる場所を含めて指定し区別したいことがある。
場所を含めた指定を、出現 (occurrnce) という。
たとえば、$x (\lambda x .(\lambda y. (xz)))$には
変数$x$の出現は3回あり、$z$の出現は1回である。

ラムダ項$Q$のなかのラムダ項$\lambda x. M$の出現で、
この$M$の出現を$\lambda x$の作用域 (scope) という。

ラムダ項$Q$のなかの変数$x$の出現が束縛された (bound) 出現であるとは、
その出現が$\lambda x$の作用域か$\lambda x$のなかにあること。
束縛されていない変数の出現を自由な (free) 出現という。

すべての変数の出現が束縛されているラムダ項を閉項 (closed term) という。


\section{閉項と証明図}


\section{証明図の正規化定理のインフォーマルな紹介}

補題ok性: 補題を使って証明していいのは、そのとき直接証明が可能であるから。

このことをわれわれは、日常の数学で用いている。

%変換

%シミュレーションレベルでの補題ok性

\begin{theo}[正規化定理]
 どんな証明図も正規な証明図に、決められたアルゴリズムで書き換えられる。
\end{theo}

\section{証明図の書き方}


\section{正規な証明図の性質、ありがたみ}
\end{document}

