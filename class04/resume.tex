\documentclass[fleqn]{jsarticle}
\title{論理学からはじめる数学: 第4講}
\author{川井新}
\usepackage{amscd}
\usepackage{amsmath}
\usepackage{amsthm}
\usepackage{mathtools}
\usepackage{array}
\usepackage{bussproofs}
\newtheorem{theo}{定理}
\newtheorem{defi}{定義}
\newtheorem{exer}{演習}

\begin{document}
\maketitle

\section{簡約と計算}

式の一部をいっそう簡単な式に置き換える操作を\textgt{簡約}という。
たとえば、$1+2$という式を$3$という式で置き換えることは簡約である。
また、$f(x)=x^2+3x+4$という関数定義があったとき、
$f(3)$に対して、
関数$f$の定義を展開して$3^2+3 \cdot 3 +4$という式で置き換えることも簡約である。
この式のなかで簡約を繰り返すと以下のようになる。

\begin{align*}
  f(1+2) &= (1+2)^2 + 3 \cdot (1+2) + 4\\
         &= 3^2 + 3 \cdot 3 + 4\\
         &= 9 + 9 + 4\\
         &= 22
\end{align*}

一般に簡約を繰り返すことで、
式の値を求めることができる。

ラムダ計算は、
関数の計算についての理論である。
そのなかで、
基本的な概念は関数であり、
すべてのものは関数として表現され、
計算は簡約により定義される。

ラムダ計算に踏み入る前にすこし予備考察をしよう。

\section{再帰}

$x$の階乗を計算する関数$f(x) = x!$は再帰的に定義できる。

\begin{gather*}
  fact(0) = 1\\
  fact(x+1) = x \cdot fact(x)
\end{gather*}

この定義は、
そのまま$f(x)$の値を求めるために用いることができる。

\begin{align*}
  fact(3) &= 3 \cdot fact(2)\\
          &= 3 \cdot (2 \cdot fact(1))\\
          &= 3 \cdot (2 \cdot (1 \cdot fact(0)))\\
          &= 3 \cdot (2 \cdot (1 \cdot 1))\\
          &= \cdots\\
          &= 6
\end{align*}

$fact(x+1)$の定義のなかで$fact$自身を参照しているので、
このような定義を\textgt{再帰的定義}という\footnote{$fact$のばあい、$0$で値を直接、定めている。この部分を\textgt{Base case}という}。
一般的に計算可能と言われている関数\footnote{Church-Turingの定立で調べるといい。}は
すべて再帰的に定義可能である。

\section{ラムダ式}

つぎのように義される関数$twice(f,x)$を考える。
\[
  twice(f,x) = f(f(x))
\]

$twice$の最初の引数$f$はひとつの引数をもつ関数である。

\begin{align*}
  twice(fact , 3) &= fact (fact (3))\\
                  &= fact(6)\\
                  &= 720
\end{align*}

$twice$を用いて${2^2}^3$を計算するには、
$exp2(x) = 2^x$という関数$exp2$を準備し、
$twice( exp2 , 3 )$を計算する。
しかし、使い捨ての関数をそのつど定義するのも手間であり、
名前不足にもなる。
このばあい、
「$x$に$2^x$を対応させる関数」を直接に表す記法があったほうが
便利で自然である。
そのような記法が\textgt{ラムダ式}である。

$x$に$2^x$を対応させる関数は、
\[ \lambda (x) 2^x \]
というラムダ式によって表すことができる。

$twice( \lambda (x) 2^x , 3)$の値を求めよう。

\begin{align*}
  twice( \lambda (x) 2^x ) &= ( \lambda (x) 2^x ( \lambda (x) 2^x )(3))\\
                           &= ( \lambda (x) 2^x ) (2^3)\\
                           &= ( \lambda (x) 2^x ) (8)\\
                           &= 2^8\\
                           &= 256
\end{align*}

関数$add$を$add (x) = \lambda (y) x+y$と定義する。
$add$は$x$に$\lambda (y) x+y$という関数を対応させる関数である。

\begin{align*}
  (add(3))(4) &= ( \lambda (y) 3 + y)(4)\\
              &= 3 + 4\\
              &= 7
\end{align*}

一般に、二引数の関数$f(x,y)$に対して、
関数$g$を$g(x) = \lambda (y) f(x,y)$と定義する。
$g$は一引数関数であり、
引数に適用されると一引数関数が求まるようなものである。
このような関数$g$を$f$の\textgt{カリー化された関数} (curried function) という。
カリー化を用いると、
二引数の関数を、したがって、$n$引数 ($n$は二以上の自然数) の関数を
一引数の関数だけで表現できる。
したがって、
基礎概念として採用する関数は一引数の関数でよさそうである。

\section{ラムダ項}


ラムダ項の定義を思い出そう。

\begin{enumerate}
    \item 変数$x, y, z \ldots$はラムダ項である。
    \item $M$と$N$がラムダ項なら、$(MN)$はラムダ項である (「適用」) 。
    \item $M$がラムダ項で$x$が変数なら、$(\lambda x. M)$はラムダ項である (「抽象」) 。
\end{enumerate}

ラムダ項はラムダ式にしたがってつぎのように読める。

\begin{center}
  \begin{tabular}{c|c}
   ラムダ式      & ラムダ項\\ \hline
   $\lambda (x)$ & $\lambda x.$\\
   $M(N)$        & $MN$
  \end{tabular}
\end{center}


$\lambda x. 2^x$などのなかの$x$は$\sum_{i=0}^{10} i(i-1)$の$i$や
$\int_0^{\pi} \sin x dx$の$x$と同じように文字を入れ替えても
式の意味・役割は変わらない。
束縛変数を置き換えてうつる構文上の合同関係を$\alpha$-同値という。
ふたつのラムダ項$M$と$N$とが$\alpha$-同値であることを
$M =_{\alpha} N$と表す。
たとえば、$\lambda x.fxz =_{\alpha} \lambda y.fyz$である。
 
$M$内の$x$の出現にラムダ項$N$に代入したものを
$M[x \coloneqq N]$とかく。
単純なばあいには、
$M$の中の$x$をすべて$N$に置き換えれば代入を得られる。
\underline{代入は束縛関係を保つ}必要があり、
一般のばあいは以下のように帰納的に定義される。


\begin{defi}
代入を帰納的に定義する:

\begin{enumerate}
  \item $x[x \coloneqq N] \equiv N$
  \item $N[x \coloneqq N] \equiv M$ ($x$が$M$に自由に出現しないとき)
  \item $(QR)[ x \coloneqq N] \equiv (Q[x \coloneqq N] R[x \coloneqq N])$
  \item $( \lambda y.Q)[x \coloneqq N] \equiv \lambda y.Q[x \coloneqq N ]$ ($y$が$N$に自由に出現しないとき)
  \item $(\lambda y.Q)[x \coloneqq N] \equiv \lambda z.Q[y \coloneqq z][x \coloneqq N]$ (ここで$z$は$N(\lambda xy.Q)$に出現しない変数)
\end{enumerate}
\end{defi}

\section{$\beta$-簡約}

ラムダ式$(\lambda (x) 2^x)(3)$は、
$2^3$に等しいと考えらて置き換えられ、
さらに簡単な式である$8$に置き換えられた。
ラムダ項についてもラムダ計算では$(\lambda x.M)N$というラムダ項を
ラムダ項$M[x \coloneqq N]$に等しいと考える。
この書き換えを\textgt{$\beta$-簡約} ($\beta$-reduction) という。

$\beta$-簡約は、
つぎの四つの規則から導かれるラムダ項の間の二項関係として、
帰納的に定義できる。

\begin{center}
  \begin{tabular}{cccc}
   \multicolumn{3}{c}{
   \AxiomC{}
   \RightLabel{Beta}
   \UnaryInfC{$(\lambda x.M)N \to_{1} M[x \coloneqq N]$}
   \DisplayProof}\\\\
   \AxiomC{$M \to_{1} M'$}
   \RightLabel{AppL}
   \UnaryInfC{$MN \to_{1} M'N$}
   \DisplayProof &
   \AxiomC{$N \to_{1} N'$}
   \RightLabel{AppR}
   \UnaryInfC{$MN \to_{1} MN'$}
   \DisplayProof &
   \AxiomC{$M \to_{1} M'$}
   \RightLabel{Lam}
   \UnaryInfC{$\lambda x.M \to_{1} \lambda x.M'$}
   \DisplayProof
  \end{tabular}
\end{center}

以上の規則にしたがって$M \to_{1} N$が導かれるとき、$M \to_{1 \beta} N$と書く。

簡約の具体例を確認しよう。

\begin{align*}
  (\lambda x.x)(\lambda y.y) &\to_{1 \beta} (\lambda y.y)(\lambda y.y)\\
                            &\to_{1 \beta} \lambda y.y
\end{align*}

この列の右端の$\lambda y.y$のように
これ以上$\beta$-簡約できないラムダ項を$\beta$-正規形という。

$\beta$-簡約、$\beta$-正規形をそれぞれ簡単に簡約、正規形ということがある。

$\beta$-簡約で連なるラムダ項の列を
\textgt{$\beta$-簡約列} ($\beta$-reduction sequence) という。

\[M_1 \to_{1 \beta} M_2 \to_{1 \beta} M_3 \to_{1 \beta} \cdots \to_{1 \beta} M \]
うえの$\beta$-簡約列のように、
$M_1$と$M$とを結ぶラムダ項の列で隣り合うラムダ項間に
二項関係$\to_{1 \beta}$が成立しているとき、
$M_1 \to_{\beta} M$と書く。



\begin{exer}
つぎのラムダ項を求めよ。

\begin{enumerate}
  \item $( \lambda y.x (\lambda x.x ))[x \coloneqq ( \lambda x.xy)]$
  \item $(y ( \lambda z.zy))[y \coloneqq ( \lambda x.xz)]$
\end{enumerate}
\end{exer}

\begin{exer}
つぎのラムダ項を$\beta$-簡約してその性質を考えよ。

\begin{enumerate}
  \item $( \lambda x.xx)( \lambda x.xx)$
  \item $( \lambda fx.x)MN$
  \item $( \lambda fx.f(f(f(fx))))MN$
  \item $( \lambda nfx. f(nfx))( \lambda fx.f(f(f(f(x)))))$
\end{enumerate}

\end{exer}

\begin{thebibliography}{99}
  \bibitem{LandC} 萩谷・西崎『論理と計算のしくみ』、岩波、2007年
  \bibitem{Genron} 古森・小野『現代数理論理学序説』、日本評論、2010年
  \bibitem{} 
\end{thebibliography}

\end{document}
